% ****************************************************************************************************
% classicthesis-config.tex 
% formerly known as loadpackages.sty, classicthesis-ldpkg.sty, and classicthesis-preamble.sty 
% Use it at the beginning of your ClassicThesis.tex, or as a LaTeX Preamble 
% in your ClassicThesis.{tex,lyx} with % ****************************************************************************************************
% classicthesis-config.tex 
% formerly known as loadpackages.sty, classicthesis-ldpkg.sty, and classicthesis-preamble.sty 
% Use it at the beginning of your ClassicThesis.tex, or as a LaTeX Preamble 
% in your ClassicThesis.{tex,lyx} with % ****************************************************************************************************
% classicthesis-config.tex 
% formerly known as loadpackages.sty, classicthesis-ldpkg.sty, and classicthesis-preamble.sty 
% Use it at the beginning of your ClassicThesis.tex, or as a LaTeX Preamble 
% in your ClassicThesis.{tex,lyx} with % ****************************************************************************************************
% classicthesis-config.tex 
% formerly known as loadpackages.sty, classicthesis-ldpkg.sty, and classicthesis-preamble.sty 
% Use it at the beginning of your ClassicThesis.tex, or as a LaTeX Preamble 
% in your ClassicThesis.{tex,lyx} with \input{classicthesis-config}
% ****************************************************************************************************  
% If you like the classicthesis, then I would appreciate a postcard. 
% My address can be found in the file ClassicThesis.pdf. A collection 
% of the postcards I received so far is available online at 
% http://postcards.miede.de
% ****************************************************************************************************


% ClassicThesis
\PassOptionsToPackage{eulerchapternumbers, listings,
					  pdfspacing, beramono,
					  eulermath,parts}{classicthesis}
\usepackage{classicthesis}                                       
% ********************************************************************
% Available options for classicthesis.sty 
% (see ClassicThesis.pdf for more information):
% drafting
% parts nochapters linedheaders
% eulerchapternumbers beramono eulermath pdfspacing minionprospacing
% tocaligned dottedtoc manychapters
% listings floatperchapter subfig
% ********************************************************************


% Personal data commands
\newcommand{\myTitle}{Notes on Data Structures and Algorithms in Python\xspace}
\newcommand{\myName}{Omar Chehaimi\xspace}
\newcommand{\myLocation}{Modena\xspace}
%\newcommand{\mySubtitle}{An Homage to The Elements of Typographic Style\xspace}
%\newcommand{\myVersion}{version 4.2\xspace}
%\newcommand{\myTime}{September 2015\xspace}
%\newcommand{\myDegree}{Doktor-Ingenieur (Dr.-Ing.)\xspace}
%\newcommand{\myProf}{Put name here\xspace}
%\newcommand{\myOtherProf}{Put name here\xspace}
%\newcommand{\mySupervisor}{Put name here\xspace}
%\newcommand{\myFaculty}{Put data here\xspace}
%\newcommand{\myDepartment}{Put data here\xspace}
%\newcommand{\myUni}{Put data here\xspace}


% Setup, finetuning, and useful commands
\newcounter{dummy} % necessary for correct hyperlinks (to index, bib, etc.)
\newlength{\abcd} % for ab..z string length calculation
\providecommand{\mLyX}{L\kern-.1667em\lower.25em\hbox{Y}\kern-.125emX\@}
\newcommand{\ie}{i.\,e.}
\newcommand{\Ie}{I.\,e.}
\newcommand{\eg}{e.\,g.}
\newcommand{\Eg}{E.\,g.} 


% Some general useful packages
\PassOptionsToPackage{utf8}{inputenc}
	\usepackage{inputenc}
\usepackage[english]{babel} % English                   
\usepackage[T1]{fontenc}
\usepackage{csquotes}
\usepackage{subcaption}
\usepackage{amsmath,amssymb} % Math
\usepackage{textcomp} % fix warning with missing font shapes
\usepackage{scrhack} % fix warnings when using KOMA with listings package          
\usepackage{xspace} % to get the spacing after macros right  
\usepackage{mparhack} % get marginpar right
\usepackage{fixltx2e} % fixes some LaTeX stuff --> since 2015 in the LaTeX kernel (see below)
\PassOptionsToPackage{printonlyused,smaller}{acronym} 
    \usepackage{acronym}
    \renewcommand*{\aclabelfont}[1]{\acsfont{#1}}
\usepackage{indentfirst}
\usepackage{chngpage}
\usepackage{calc}
\PassOptionsToPackage{pdftex}{graphicx}
    \usepackage{graphicx} 
\usepackage{shapepar}
\usepackage{pifont}
\usepackage{ccicons}


% Bibliography
\PassOptionsToPackage{
    %backend=biber,
	backend=bibtex8, 
	%bibencoding=ascii,
	%language=auto,
	%style=numeric-comp,
    %style=authoryear-comp, % Author 1999, 2010
    %bibstyle=authoryear,dashed=false, % dashed: substitute rep. author with ---
    %sorting=nyt, % name, year, title
    sorting=none
    %maxbibnames=10, % default: 3, et al.
    %backref=true,%
    %natbib=true % natbib compatibility mode (\citep and \citet still work)
}{biblatex}
    \usepackage{biblatex}


% Setup floats: tables, (sub)figures, and captions
\usepackage{float} % For forcing the position of a picture
\usepackage{wrapfig} % For wrapping the text to a picture
\usepackage{color, colortbl} % Coloring boxes in a table
    \definecolor{LightCyan}{rgb}{0.88,1,1}
    \definecolor{Maroon}{cmyk}{0, 0.87, 0.68, 0.32}

% Box sourrounding numbers
\usepackage{tikz}
\newcommand\mybox[3][]{%
    \tikz[anchor=base,baseline]\node[inner sep=2pt,draw=#2,#1]{$	 \displaystyle#3\mathstrut$};}
\colorlet{mycol}{black}

% Arrows in tables
\usetikzlibrary{tikzmark}
\usetikzlibrary{arrows.meta}
\usetikzlibrary{calc, arrows}
\usetikzlibrary{positioning}
\usetikzlibrary{trees, shapes.misc}
\usetikzlibrary{matrix, fit, shapes.multipart}
\usetikzlibrary{automata, shapes.geometric}
\usetikzlibrary{decorations.pathreplacing, bending, chains}

% Array
\usepackage{array}

% Forest
\usepackage{forest}

% Code listings and pseudocode
\usepackage{listings}
\newcommand{\mail}[1]{\href{mailto:#1}{\texttt{#1}}}
\definecolor{lightergray}{gray}{0.99}
\definecolor{deepblue}{rgb}{0,0,0.5}
\definecolor{deepred}{rgb}{0.6,0,0}
\definecolor{deepgreen}{rgb}{0,0.5,0}
\lstset{language=python,
        keywordstyle=\color{RoyalBlue},
        basicstyle=\small\ttfamily,
        commentstyle=\color{Emerald}\ttfamily,
		morekeywords={self},
		keywordstyle=\color{deepblue},
		emph={__init__},
		emphstyle=\color{deepred},
		stringstyle=\color{deepgreen},
        numberstyle=\scriptsize,
        showstringspaces=false,
        breaklines=true,
        %frame=lines, % No lines to limit the code
        backgroundcolor=\color{lightergray},
        flexiblecolumns=true,
        escapeinside={£*}{*£},
        firstnumber=last,
        numberfirstline=true,
		numbers=left,
}
\usepackage[ruled, vlined, linesnumbered]{algorithm2e} % Pseudocode
\usepackage{chngcntr} % For counting the listing within the chapter.				% The command is right after begin document (\counterwithin{lstlisting}{chapter})  


% PDFLaTeX, hyperreferences and citation backreferences
% Using PDFLaTeX
\PassOptionsToPackage{pdftex,hyperfootnotes=false,pdfpagelabels}{hyperref}
    \usepackage{hyperref}  % backref linktocpage pagebackref
\pdfcompresslevel=9
\pdfadjustspacing=1 


% Hyperreferences
\hypersetup{
    colorlinks=true, linktocpage=true, pdfstartpage=3, 
    pdfstartview=FitV,
    % Uncomment the following line if you want to have 
    % black links (e.g., for printing)
    %colorlinks=false, linktocpage=false, pdfstartpage=3, 
    %pdfstartview=FitV, pdfborder={0 0 0},%
    breaklinks=true, pdfpagemode=UseNone, pageanchor=true, 
    pdfpagemode=UseOutlines, plainpages=false, bookmarksnumbered, 
    bookmarksopen=true, bookmarksopenlevel=1,hypertexnames=true, 
    pdfhighlight=/O, %nesting=true, %frenchlinks,
    urlcolor=RoyalBlue, linkcolor=RoyalBlue, citecolor=RoyalBlue,
    hyperfootnotes=false, pdfpagelabels,
    pdftitle={\myTitle},
    pdfauthor={\copyright \myName},
    pdfsubject={Data Structures and Algorithms},
    pdfkeywords={computer science, algorithms, data structures},
    pdfcreator={pdfLaTeX},
    pdfproducer={LaTeX with hyperref and classicthesis}
}


% Write in log all the files present in this project
\listfiles


% Changing the text area
\areaset[current]{370pt}{784pt}

% Using different fonts
%\usepackage[oldstylenums]{kpfonts} % oldstyle notextcomp
%\usepackage[osf]{libertine}
%\usepackage[light,condensed,math]{iwona}
\renewcommand{\sfdefault}{iwona}
%\usepackage{lmodern} % <-- no osf support :-(
%\usepackage{cfr-lm} % 
%\usepackage[urw-garamond]{mathdesign} <-- no osf support :-(
%\usepackage[default,osfigures]{opensans} % scale=0.95 
%\usepackage[sfdefault]{FiraSans}


% Making bigger the font for parts, chapters, sections, and subsections
% Part
\titleformat{\part}[display]
	{\normalfont\centering\Huge}% Huge
    {\thispagestyle{empty}\partname~\MakeTextUppercase{\thepart}}{1em}%
    {\color{Maroon}\spacedallcaps}

% Chapter
\RequirePackage{ifthen}
% In this way a newfont \chapterNumber can be defined
\let\chapterNumber\undefined
\ifthenelse{\boolean{@eulerchapternumbers}}
	{\newfont{\chapterNumber}{eurb10 scaled 6000}}

\titleformat{\chapter}[block]
	{\normalfont\Large\sffamily}{\mbox{}\oldmarginpar{\vspace*{-3\baselineskip}\color{halfgray}\chapterNumber\thechapter}}{0pt}%
	{\raggedright\spacedallcaps}[\normalfont\Large\sffamily\vspace*{.8\baselineskip}]%

% Sections
\titleformat{\section}
	{\normalfont\Large\sffamily\color{Maroon}}{\textsc{\MakeTextLowercase{\thesection}}}{1em}{\spacedlowsmallcaps}
% Subsections
\titleformat{\subsection}
	{\normalfont\sffamily\bfseries\color{Maroon}}{\textsc{\MakeTextLowercase{\thesubsection}}}{1em}{\normalsize}
% Subsubsections
\titleformat{\subsubsection}
	{\normalfont\sffamily\bfseries\itshape\color{Maroon}}{\textsc{\MakeTextLowercase{\thesubsubsection}}}{1em}{\normalsize}      
% Paragraphs
\titleformat{\paragraph}[runin]
	{\normalfont\normalsize\sffamily\bfseries\color{Maroon}}{\theparagraph}{0pt}{\spacedlowsmallcaps} 


% Caption in bold font and with the right position 
% in case of too long text (hang format) 
\captionsetup{format=hang, font=small, labelfont=bf}
\captionsetup[table]{skip=\medskipamount}
% ****************************************************************************************************  
% If you like the classicthesis, then I would appreciate a postcard. 
% My address can be found in the file ClassicThesis.pdf. A collection 
% of the postcards I received so far is available online at 
% http://postcards.miede.de
% ****************************************************************************************************


% ClassicThesis
\PassOptionsToPackage{eulerchapternumbers, listings,
					  pdfspacing, beramono,
					  eulermath,parts}{classicthesis}
\usepackage{classicthesis}                                       
% ********************************************************************
% Available options for classicthesis.sty 
% (see ClassicThesis.pdf for more information):
% drafting
% parts nochapters linedheaders
% eulerchapternumbers beramono eulermath pdfspacing minionprospacing
% tocaligned dottedtoc manychapters
% listings floatperchapter subfig
% ********************************************************************


% Personal data commands
\newcommand{\myTitle}{Notes on Data Structures and Algorithms in Python\xspace}
\newcommand{\myName}{Omar Chehaimi\xspace}
\newcommand{\myLocation}{Modena\xspace}
%\newcommand{\mySubtitle}{An Homage to The Elements of Typographic Style\xspace}
%\newcommand{\myVersion}{version 4.2\xspace}
%\newcommand{\myTime}{September 2015\xspace}
%\newcommand{\myDegree}{Doktor-Ingenieur (Dr.-Ing.)\xspace}
%\newcommand{\myProf}{Put name here\xspace}
%\newcommand{\myOtherProf}{Put name here\xspace}
%\newcommand{\mySupervisor}{Put name here\xspace}
%\newcommand{\myFaculty}{Put data here\xspace}
%\newcommand{\myDepartment}{Put data here\xspace}
%\newcommand{\myUni}{Put data here\xspace}


% Setup, finetuning, and useful commands
\newcounter{dummy} % necessary for correct hyperlinks (to index, bib, etc.)
\newlength{\abcd} % for ab..z string length calculation
\providecommand{\mLyX}{L\kern-.1667em\lower.25em\hbox{Y}\kern-.125emX\@}
\newcommand{\ie}{i.\,e.}
\newcommand{\Ie}{I.\,e.}
\newcommand{\eg}{e.\,g.}
\newcommand{\Eg}{E.\,g.} 


% Some general useful packages
\PassOptionsToPackage{utf8}{inputenc}
	\usepackage{inputenc}
\usepackage[english]{babel} % English                   
\usepackage[T1]{fontenc}
\usepackage{csquotes}
\usepackage{subcaption}
\usepackage{amsmath,amssymb} % Math
\usepackage{textcomp} % fix warning with missing font shapes
\usepackage{scrhack} % fix warnings when using KOMA with listings package          
\usepackage{xspace} % to get the spacing after macros right  
\usepackage{mparhack} % get marginpar right
\usepackage{fixltx2e} % fixes some LaTeX stuff --> since 2015 in the LaTeX kernel (see below)
\PassOptionsToPackage{printonlyused,smaller}{acronym} 
    \usepackage{acronym}
    \renewcommand*{\aclabelfont}[1]{\acsfont{#1}}
\usepackage{indentfirst}
\usepackage{chngpage}
\usepackage{calc}
\PassOptionsToPackage{pdftex}{graphicx}
    \usepackage{graphicx} 
\usepackage{shapepar}
\usepackage{pifont}
\usepackage{ccicons}


% Bibliography
\PassOptionsToPackage{
    %backend=biber,
	backend=bibtex8, 
	%bibencoding=ascii,
	%language=auto,
	%style=numeric-comp,
    %style=authoryear-comp, % Author 1999, 2010
    %bibstyle=authoryear,dashed=false, % dashed: substitute rep. author with ---
    %sorting=nyt, % name, year, title
    sorting=none
    %maxbibnames=10, % default: 3, et al.
    %backref=true,%
    %natbib=true % natbib compatibility mode (\citep and \citet still work)
}{biblatex}
    \usepackage{biblatex}


% Setup floats: tables, (sub)figures, and captions
\usepackage{float} % For forcing the position of a picture
\usepackage{wrapfig} % For wrapping the text to a picture
\usepackage{color, colortbl} % Coloring boxes in a table
    \definecolor{LightCyan}{rgb}{0.88,1,1}
    \definecolor{Maroon}{cmyk}{0, 0.87, 0.68, 0.32}

% Box sourrounding numbers
\usepackage{tikz}
\newcommand\mybox[3][]{%
    \tikz[anchor=base,baseline]\node[inner sep=2pt,draw=#2,#1]{$	 \displaystyle#3\mathstrut$};}
\colorlet{mycol}{black}

% Arrows in tables
\usetikzlibrary{tikzmark}
\usetikzlibrary{arrows.meta}
\usetikzlibrary{calc, arrows}
\usetikzlibrary{positioning}
\usetikzlibrary{trees, shapes.misc}
\usetikzlibrary{matrix, fit, shapes.multipart}
\usetikzlibrary{automata, shapes.geometric}
\usetikzlibrary{decorations.pathreplacing, bending, chains}

% Array
\usepackage{array}

% Forest
\usepackage{forest}

% Code listings and pseudocode
\usepackage{listings}
\newcommand{\mail}[1]{\href{mailto:#1}{\texttt{#1}}}
\definecolor{lightergray}{gray}{0.99}
\definecolor{deepblue}{rgb}{0,0,0.5}
\definecolor{deepred}{rgb}{0.6,0,0}
\definecolor{deepgreen}{rgb}{0,0.5,0}
\lstset{language=python,
        keywordstyle=\color{RoyalBlue},
        basicstyle=\small\ttfamily,
        commentstyle=\color{Emerald}\ttfamily,
		morekeywords={self},
		keywordstyle=\color{deepblue},
		emph={__init__},
		emphstyle=\color{deepred},
		stringstyle=\color{deepgreen},
        numberstyle=\scriptsize,
        showstringspaces=false,
        breaklines=true,
        %frame=lines, % No lines to limit the code
        backgroundcolor=\color{lightergray},
        flexiblecolumns=true,
        escapeinside={£*}{*£},
        firstnumber=last,
        numberfirstline=true,
		numbers=left,
}
\usepackage[ruled, vlined, linesnumbered]{algorithm2e} % Pseudocode
\usepackage{chngcntr} % For counting the listing within the chapter.				% The command is right after begin document (\counterwithin{lstlisting}{chapter})  


% PDFLaTeX, hyperreferences and citation backreferences
% Using PDFLaTeX
\PassOptionsToPackage{pdftex,hyperfootnotes=false,pdfpagelabels}{hyperref}
    \usepackage{hyperref}  % backref linktocpage pagebackref
\pdfcompresslevel=9
\pdfadjustspacing=1 


% Hyperreferences
\hypersetup{
    colorlinks=true, linktocpage=true, pdfstartpage=3, 
    pdfstartview=FitV,
    % Uncomment the following line if you want to have 
    % black links (e.g., for printing)
    %colorlinks=false, linktocpage=false, pdfstartpage=3, 
    %pdfstartview=FitV, pdfborder={0 0 0},%
    breaklinks=true, pdfpagemode=UseNone, pageanchor=true, 
    pdfpagemode=UseOutlines, plainpages=false, bookmarksnumbered, 
    bookmarksopen=true, bookmarksopenlevel=1,hypertexnames=true, 
    pdfhighlight=/O, %nesting=true, %frenchlinks,
    urlcolor=RoyalBlue, linkcolor=RoyalBlue, citecolor=RoyalBlue,
    hyperfootnotes=false, pdfpagelabels,
    pdftitle={\myTitle},
    pdfauthor={\copyright \myName},
    pdfsubject={Data Structures and Algorithms},
    pdfkeywords={computer science, algorithms, data structures},
    pdfcreator={pdfLaTeX},
    pdfproducer={LaTeX with hyperref and classicthesis}
}


% Write in log all the files present in this project
\listfiles


% Changing the text area
\areaset[current]{370pt}{784pt}

% Using different fonts
%\usepackage[oldstylenums]{kpfonts} % oldstyle notextcomp
%\usepackage[osf]{libertine}
%\usepackage[light,condensed,math]{iwona}
\renewcommand{\sfdefault}{iwona}
%\usepackage{lmodern} % <-- no osf support :-(
%\usepackage{cfr-lm} % 
%\usepackage[urw-garamond]{mathdesign} <-- no osf support :-(
%\usepackage[default,osfigures]{opensans} % scale=0.95 
%\usepackage[sfdefault]{FiraSans}


% Making bigger the font for parts, chapters, sections, and subsections
% Part
\titleformat{\part}[display]
	{\normalfont\centering\Huge}% Huge
    {\thispagestyle{empty}\partname~\MakeTextUppercase{\thepart}}{1em}%
    {\color{Maroon}\spacedallcaps}

% Chapter
\RequirePackage{ifthen}
% In this way a newfont \chapterNumber can be defined
\let\chapterNumber\undefined
\ifthenelse{\boolean{@eulerchapternumbers}}
	{\newfont{\chapterNumber}{eurb10 scaled 6000}}

\titleformat{\chapter}[block]
	{\normalfont\Large\sffamily}{\mbox{}\oldmarginpar{\vspace*{-3\baselineskip}\color{halfgray}\chapterNumber\thechapter}}{0pt}%
	{\raggedright\spacedallcaps}[\normalfont\Large\sffamily\vspace*{.8\baselineskip}]%

% Sections
\titleformat{\section}
	{\normalfont\Large\sffamily\color{Maroon}}{\textsc{\MakeTextLowercase{\thesection}}}{1em}{\spacedlowsmallcaps}
% Subsections
\titleformat{\subsection}
	{\normalfont\sffamily\bfseries\color{Maroon}}{\textsc{\MakeTextLowercase{\thesubsection}}}{1em}{\normalsize}
% Subsubsections
\titleformat{\subsubsection}
	{\normalfont\sffamily\bfseries\itshape\color{Maroon}}{\textsc{\MakeTextLowercase{\thesubsubsection}}}{1em}{\normalsize}      
% Paragraphs
\titleformat{\paragraph}[runin]
	{\normalfont\normalsize\sffamily\bfseries\color{Maroon}}{\theparagraph}{0pt}{\spacedlowsmallcaps} 


% Caption in bold font and with the right position 
% in case of too long text (hang format) 
\captionsetup{format=hang, font=small, labelfont=bf}
\captionsetup[table]{skip=\medskipamount}
% ****************************************************************************************************  
% If you like the classicthesis, then I would appreciate a postcard. 
% My address can be found in the file ClassicThesis.pdf. A collection 
% of the postcards I received so far is available online at 
% http://postcards.miede.de
% ****************************************************************************************************


% ClassicThesis
\PassOptionsToPackage{eulerchapternumbers, listings,
					  pdfspacing, beramono,
					  eulermath,parts}{classicthesis}
\usepackage{classicthesis}                                       
% ********************************************************************
% Available options for classicthesis.sty 
% (see ClassicThesis.pdf for more information):
% drafting
% parts nochapters linedheaders
% eulerchapternumbers beramono eulermath pdfspacing minionprospacing
% tocaligned dottedtoc manychapters
% listings floatperchapter subfig
% ********************************************************************


% Personal data commands
\newcommand{\myTitle}{Notes on Data Structures and Algorithms in Python\xspace}
\newcommand{\myName}{Omar Chehaimi\xspace}
\newcommand{\myLocation}{Modena\xspace}
%\newcommand{\mySubtitle}{An Homage to The Elements of Typographic Style\xspace}
%\newcommand{\myVersion}{version 4.2\xspace}
%\newcommand{\myTime}{September 2015\xspace}
%\newcommand{\myDegree}{Doktor-Ingenieur (Dr.-Ing.)\xspace}
%\newcommand{\myProf}{Put name here\xspace}
%\newcommand{\myOtherProf}{Put name here\xspace}
%\newcommand{\mySupervisor}{Put name here\xspace}
%\newcommand{\myFaculty}{Put data here\xspace}
%\newcommand{\myDepartment}{Put data here\xspace}
%\newcommand{\myUni}{Put data here\xspace}


% Setup, finetuning, and useful commands
\newcounter{dummy} % necessary for correct hyperlinks (to index, bib, etc.)
\newlength{\abcd} % for ab..z string length calculation
\providecommand{\mLyX}{L\kern-.1667em\lower.25em\hbox{Y}\kern-.125emX\@}
\newcommand{\ie}{i.\,e.}
\newcommand{\Ie}{I.\,e.}
\newcommand{\eg}{e.\,g.}
\newcommand{\Eg}{E.\,g.} 


% Some general useful packages
\PassOptionsToPackage{utf8}{inputenc}
	\usepackage{inputenc}
\usepackage[english]{babel} % English                   
\usepackage[T1]{fontenc}
\usepackage{csquotes}
\usepackage{subcaption}
\usepackage{amsmath,amssymb} % Math
\usepackage{textcomp} % fix warning with missing font shapes
\usepackage{scrhack} % fix warnings when using KOMA with listings package          
\usepackage{xspace} % to get the spacing after macros right  
\usepackage{mparhack} % get marginpar right
\usepackage{fixltx2e} % fixes some LaTeX stuff --> since 2015 in the LaTeX kernel (see below)
\PassOptionsToPackage{printonlyused,smaller}{acronym} 
    \usepackage{acronym}
    \renewcommand*{\aclabelfont}[1]{\acsfont{#1}}
\usepackage{indentfirst}
\usepackage{chngpage}
\usepackage{calc}
\PassOptionsToPackage{pdftex}{graphicx}
    \usepackage{graphicx} 
\usepackage{shapepar}
\usepackage{pifont}
\usepackage{ccicons}


% Bibliography
\PassOptionsToPackage{
    %backend=biber,
	backend=bibtex8, 
	%bibencoding=ascii,
	%language=auto,
	%style=numeric-comp,
    %style=authoryear-comp, % Author 1999, 2010
    %bibstyle=authoryear,dashed=false, % dashed: substitute rep. author with ---
    %sorting=nyt, % name, year, title
    sorting=none
    %maxbibnames=10, % default: 3, et al.
    %backref=true,%
    %natbib=true % natbib compatibility mode (\citep and \citet still work)
}{biblatex}
    \usepackage{biblatex}


% Setup floats: tables, (sub)figures, and captions
\usepackage{float} % For forcing the position of a picture
\usepackage{wrapfig} % For wrapping the text to a picture
\usepackage{color, colortbl} % Coloring boxes in a table
    \definecolor{LightCyan}{rgb}{0.88,1,1}
    \definecolor{Maroon}{cmyk}{0, 0.87, 0.68, 0.32}

% Box sourrounding numbers
\usepackage{tikz}
\newcommand\mybox[3][]{%
    \tikz[anchor=base,baseline]\node[inner sep=2pt,draw=#2,#1]{$	 \displaystyle#3\mathstrut$};}
\colorlet{mycol}{black}

% Arrows in tables
\usetikzlibrary{tikzmark}
\usetikzlibrary{arrows.meta}
\usetikzlibrary{calc, arrows}
\usetikzlibrary{positioning}
\usetikzlibrary{trees, shapes.misc}
\usetikzlibrary{matrix, fit, shapes.multipart}
\usetikzlibrary{automata, shapes.geometric}
\usetikzlibrary{decorations.pathreplacing, bending, chains}

% Array
\usepackage{array}

% Forest
\usepackage{forest}

% Code listings and pseudocode
\usepackage{listings}
\newcommand{\mail}[1]{\href{mailto:#1}{\texttt{#1}}}
\definecolor{lightergray}{gray}{0.99}
\definecolor{deepblue}{rgb}{0,0,0.5}
\definecolor{deepred}{rgb}{0.6,0,0}
\definecolor{deepgreen}{rgb}{0,0.5,0}
\lstset{language=python,
        keywordstyle=\color{RoyalBlue},
        basicstyle=\small\ttfamily,
        commentstyle=\color{Emerald}\ttfamily,
		morekeywords={self},
		keywordstyle=\color{deepblue},
		emph={__init__},
		emphstyle=\color{deepred},
		stringstyle=\color{deepgreen},
        numberstyle=\scriptsize,
        showstringspaces=false,
        breaklines=true,
        %frame=lines, % No lines to limit the code
        backgroundcolor=\color{lightergray},
        flexiblecolumns=true,
        escapeinside={£*}{*£},
        firstnumber=last,
        numberfirstline=true,
		numbers=left,
}
\usepackage[ruled, vlined, linesnumbered]{algorithm2e} % Pseudocode
\usepackage{chngcntr} % For counting the listing within the chapter.				% The command is right after begin document (\counterwithin{lstlisting}{chapter})  


% PDFLaTeX, hyperreferences and citation backreferences
% Using PDFLaTeX
\PassOptionsToPackage{pdftex,hyperfootnotes=false,pdfpagelabels}{hyperref}
    \usepackage{hyperref}  % backref linktocpage pagebackref
\pdfcompresslevel=9
\pdfadjustspacing=1 


% Hyperreferences
\hypersetup{
    colorlinks=true, linktocpage=true, pdfstartpage=3, 
    pdfstartview=FitV,
    % Uncomment the following line if you want to have 
    % black links (e.g., for printing)
    %colorlinks=false, linktocpage=false, pdfstartpage=3, 
    %pdfstartview=FitV, pdfborder={0 0 0},%
    breaklinks=true, pdfpagemode=UseNone, pageanchor=true, 
    pdfpagemode=UseOutlines, plainpages=false, bookmarksnumbered, 
    bookmarksopen=true, bookmarksopenlevel=1,hypertexnames=true, 
    pdfhighlight=/O, %nesting=true, %frenchlinks,
    urlcolor=RoyalBlue, linkcolor=RoyalBlue, citecolor=RoyalBlue,
    hyperfootnotes=false, pdfpagelabels,
    pdftitle={\myTitle},
    pdfauthor={\copyright \myName},
    pdfsubject={Data Structures and Algorithms},
    pdfkeywords={computer science, algorithms, data structures},
    pdfcreator={pdfLaTeX},
    pdfproducer={LaTeX with hyperref and classicthesis}
}


% Write in log all the files present in this project
\listfiles


% Changing the text area
\areaset[current]{370pt}{784pt}

% Using different fonts
%\usepackage[oldstylenums]{kpfonts} % oldstyle notextcomp
%\usepackage[osf]{libertine}
%\usepackage[light,condensed,math]{iwona}
\renewcommand{\sfdefault}{iwona}
%\usepackage{lmodern} % <-- no osf support :-(
%\usepackage{cfr-lm} % 
%\usepackage[urw-garamond]{mathdesign} <-- no osf support :-(
%\usepackage[default,osfigures]{opensans} % scale=0.95 
%\usepackage[sfdefault]{FiraSans}


% Making bigger the font for parts, chapters, sections, and subsections
% Part
\titleformat{\part}[display]
	{\normalfont\centering\Huge}% Huge
    {\thispagestyle{empty}\partname~\MakeTextUppercase{\thepart}}{1em}%
    {\color{Maroon}\spacedallcaps}

% Chapter
\RequirePackage{ifthen}
% In this way a newfont \chapterNumber can be defined
\let\chapterNumber\undefined
\ifthenelse{\boolean{@eulerchapternumbers}}
	{\newfont{\chapterNumber}{eurb10 scaled 6000}}

\titleformat{\chapter}[block]
	{\normalfont\Large\sffamily}{\mbox{}\oldmarginpar{\vspace*{-3\baselineskip}\color{halfgray}\chapterNumber\thechapter}}{0pt}%
	{\raggedright\spacedallcaps}[\normalfont\Large\sffamily\vspace*{.8\baselineskip}]%

% Sections
\titleformat{\section}
	{\normalfont\Large\sffamily\color{Maroon}}{\textsc{\MakeTextLowercase{\thesection}}}{1em}{\spacedlowsmallcaps}
% Subsections
\titleformat{\subsection}
	{\normalfont\sffamily\bfseries\color{Maroon}}{\textsc{\MakeTextLowercase{\thesubsection}}}{1em}{\normalsize}
% Subsubsections
\titleformat{\subsubsection}
	{\normalfont\sffamily\bfseries\itshape\color{Maroon}}{\textsc{\MakeTextLowercase{\thesubsubsection}}}{1em}{\normalsize}      
% Paragraphs
\titleformat{\paragraph}[runin]
	{\normalfont\normalsize\sffamily\bfseries\color{Maroon}}{\theparagraph}{0pt}{\spacedlowsmallcaps} 


% Caption in bold font and with the right position 
% in case of too long text (hang format) 
\captionsetup{format=hang, font=small, labelfont=bf}
\captionsetup[table]{skip=\medskipamount}
% ****************************************************************************************************  
% If you like the classicthesis, then I would appreciate a postcard. 
% My address can be found in the file ClassicThesis.pdf. A collection 
% of the postcards I received so far is available online at 
% http://postcards.miede.de
% ****************************************************************************************************


% ClassicThesis
\PassOptionsToPackage{eulerchapternumbers, listings,
					  pdfspacing, beramono,
					  eulermath,parts}{classicthesis}
\usepackage{classicthesis}                                       
% ********************************************************************
% Available options for classicthesis.sty 
% (see ClassicThesis.pdf for more information):
% drafting
% parts nochapters linedheaders
% eulerchapternumbers beramono eulermath pdfspacing minionprospacing
% tocaligned dottedtoc manychapters
% listings floatperchapter subfig
% ********************************************************************


% Personal data commands
\newcommand{\myTitle}{Notes on Data Structures and Algorithms in Python\xspace}
\newcommand{\myName}{Omar Chehaimi\xspace}
\newcommand{\myLocation}{Modena\xspace}
%\newcommand{\mySubtitle}{An Homage to The Elements of Typographic Style\xspace}
%\newcommand{\myVersion}{version 4.2\xspace}
%\newcommand{\myTime}{September 2015\xspace}
%\newcommand{\myDegree}{Doktor-Ingenieur (Dr.-Ing.)\xspace}
%\newcommand{\myProf}{Put name here\xspace}
%\newcommand{\myOtherProf}{Put name here\xspace}
%\newcommand{\mySupervisor}{Put name here\xspace}
%\newcommand{\myFaculty}{Put data here\xspace}
%\newcommand{\myDepartment}{Put data here\xspace}
%\newcommand{\myUni}{Put data here\xspace}


% Setup, finetuning, and useful commands
\newcounter{dummy} % necessary for correct hyperlinks (to index, bib, etc.)
\newlength{\abcd} % for ab..z string length calculation
\providecommand{\mLyX}{L\kern-.1667em\lower.25em\hbox{Y}\kern-.125emX\@}
\newcommand{\ie}{i.\,e.}
\newcommand{\Ie}{I.\,e.}
\newcommand{\eg}{e.\,g.}
\newcommand{\Eg}{E.\,g.} 


% Some general useful packages
\PassOptionsToPackage{utf8}{inputenc}
	\usepackage{inputenc}
\usepackage[english]{babel} % English                   
\usepackage[T1]{fontenc}
\usepackage{csquotes}
\usepackage{subcaption}
\usepackage{amsmath,amssymb} % Math
\usepackage{textcomp} % fix warning with missing font shapes
\usepackage{scrhack} % fix warnings when using KOMA with listings package          
\usepackage{xspace} % to get the spacing after macros right  
\usepackage{mparhack} % get marginpar right
\usepackage{fixltx2e} % fixes some LaTeX stuff --> since 2015 in the LaTeX kernel (see below)
\PassOptionsToPackage{printonlyused,smaller}{acronym} 
    \usepackage{acronym}
    \renewcommand*{\aclabelfont}[1]{\acsfont{#1}}
\usepackage{indentfirst}
\usepackage{chngpage}
\usepackage{calc}
\PassOptionsToPackage{pdftex}{graphicx}
    \usepackage{graphicx} 
\usepackage{shapepar}
\usepackage{pifont}
\usepackage{ccicons}


% Bibliography
\PassOptionsToPackage{
    %backend=biber,
	backend=bibtex8, 
	%bibencoding=ascii,
	%language=auto,
	%style=numeric-comp,
    %style=authoryear-comp, % Author 1999, 2010
    %bibstyle=authoryear,dashed=false, % dashed: substitute rep. author with ---
    %sorting=nyt, % name, year, title
    sorting=none
    %maxbibnames=10, % default: 3, et al.
    %backref=true,%
    %natbib=true % natbib compatibility mode (\citep and \citet still work)
}{biblatex}
    \usepackage{biblatex}


% Setup floats: tables, (sub)figures, and captions
\usepackage{float} % For forcing the position of a picture
\usepackage{wrapfig} % For wrapping the text to a picture
\usepackage{color, colortbl} % Coloring boxes in a table
    \definecolor{LightCyan}{rgb}{0.88,1,1}
    \definecolor{Maroon}{cmyk}{0, 0.87, 0.68, 0.32}

% Box sourrounding numbers
\usepackage{tikz}
\newcommand\mybox[3][]{%
    \tikz[anchor=base,baseline]\node[inner sep=2pt,draw=#2,#1]{$	 \displaystyle#3\mathstrut$};}
\colorlet{mycol}{black}

% Arrows in tables
\usetikzlibrary{tikzmark}
\usetikzlibrary{arrows.meta}
\usetikzlibrary{calc, arrows}
\usetikzlibrary{positioning}

% Code listings and pseudocode
\usepackage{listings}
\newcommand{\mail}[1]{\href{mailto:#1}{\texttt{#1}}}
\definecolor{lightergray}{gray}{0.99}
\definecolor{deepblue}{rgb}{0,0,0.5}
\definecolor{deepred}{rgb}{0.6,0,0}
\definecolor{deepgreen}{rgb}{0,0.5,0}
\lstset{language=python,
        keywordstyle=\color{RoyalBlue},
        basicstyle=\small\ttfamily,
        commentstyle=\color{Emerald}\ttfamily,
		morekeywords={self},
		keywordstyle=\color{deepblue},
		emph={__init__},
		emphstyle=\color{deepred},
		stringstyle=\color{deepgreen},
        numberstyle=\scriptsize,
        showstringspaces=false,
        breaklines=true,
        %frame=lines, % No lines to limit the code
        backgroundcolor=\color{lightergray},
        flexiblecolumns=true,
        escapeinside={£*}{*£},
        firstnumber=last,
        numberfirstline=true,
		numbers=left,
}
\usepackage[ruled, vlined, linesnumbered]{algorithm2e} % Pseudocode
\usepackage{chngcntr} % For counting the listing within the chapter.				% The command is right after begin document (\counterwithin{lstlisting}{chapter})  


% PDFLaTeX, hyperreferences and citation backreferences
% Using PDFLaTeX
\PassOptionsToPackage{pdftex,hyperfootnotes=false,pdfpagelabels}{hyperref}
    \usepackage{hyperref}  % backref linktocpage pagebackref
\pdfcompresslevel=9
\pdfadjustspacing=1 


% Hyperreferences
\hypersetup{
    colorlinks=true, linktocpage=true, pdfstartpage=3, 
    pdfstartview=FitV,
    % Uncomment the following line if you want to have 
    % black links (e.g., for printing)
    %colorlinks=false, linktocpage=false, pdfstartpage=3, 
    %pdfstartview=FitV, pdfborder={0 0 0},%
    breaklinks=true, pdfpagemode=UseNone, pageanchor=true, 
    pdfpagemode=UseOutlines, plainpages=false, bookmarksnumbered, 
    bookmarksopen=true, bookmarksopenlevel=1,hypertexnames=true, 
    pdfhighlight=/O, %nesting=true, %frenchlinks,
    urlcolor=RoyalBlue, linkcolor=RoyalBlue, citecolor=RoyalBlue,
    hyperfootnotes=false, pdfpagelabels,
    pdftitle={\myTitle},
    pdfauthor={\copyright \myName},
    pdfsubject={Data Structures and Algorithms},
    pdfkeywords={computer science, algorithms, data structures},
    pdfcreator={pdfLaTeX},
    pdfproducer={LaTeX with hyperref and classicthesis}
}


% Write in log all the files present in this project
\listfiles


% Changing the text area
\areaset[current]{370pt}{784pt}

% Using different fonts
%\usepackage[oldstylenums]{kpfonts} % oldstyle notextcomp
%\usepackage[osf]{libertine}
%\usepackage[light,condensed,math]{iwona}
\renewcommand{\sfdefault}{iwona}
%\usepackage{lmodern} % <-- no osf support :-(
%\usepackage{cfr-lm} % 
%\usepackage[urw-garamond]{mathdesign} <-- no osf support :-(
%\usepackage[default,osfigures]{opensans} % scale=0.95 
%\usepackage[sfdefault]{FiraSans}


% Making bigger the font for parts, chapters, sections, and subsections
% Part
\titleformat{\part}[display]
	{\normalfont\centering\Huge}% Huge
    {\thispagestyle{empty}\partname~\MakeTextUppercase{\thepart}}{1em}%
    {\color{Maroon}\spacedallcaps}

% Chapter
\RequirePackage{ifthen}
% In this way a newfont \chapterNumber can be defined
\let\chapterNumber\undefined
\ifthenelse{\boolean{@eulerchapternumbers}}
	{\newfont{\chapterNumber}{eurb10 scaled 6000}}

\titleformat{\chapter}[block]
	{\normalfont\Large\sffamily}{\mbox{}\oldmarginpar{\vspace*{-3\baselineskip}\color{halfgray}\chapterNumber\thechapter}}{0pt}%
	{\raggedright\spacedallcaps}[\normalfont\Large\sffamily\vspace*{.8\baselineskip}]%

% Sections
\titleformat{\section}
	{\normalfont\Large\sffamily\color{Maroon}}{\textsc{\MakeTextLowercase{\thesection}}}{1em}{\spacedlowsmallcaps}
% Subsections
\titleformat{\subsection}
	{\normalfont\sffamily\bfseries\color{Maroon}}{\textsc{\MakeTextLowercase{\thesubsection}}}{1em}{\normalsize}
% Subsubsections
\titleformat{\subsubsection}
	{\normalfont\sffamily\bfseries\itshape\color{Maroon}}{\textsc{\MakeTextLowercase{\thesubsubsection}}}{1em}{\normalsize}      
% Paragraphs
\titleformat{\paragraph}[runin]
	{\normalfont\normalsize\sffamily\bfseries\color{Maroon}}{\theparagraph}{0pt}{\spacedlowsmallcaps} 


% Caption in bold font and with the right position 
% in case of too long text (hang format) 
\captionsetup{format=hang, font=small, labelfont=bf}
\captionsetup[table]{skip=\medskipamount}