% Boxes for definitions and examples
\usepackage[framemethod=TikZ]{mdframed}
\usepackage{amsthm}
\usepackage{thmtools}


\declaretheoremstyle[
    headfont=\bfseries, 
    notebraces={}{},
    bodyfont=\normalfont\itshape,
    headpunct={},
    postheadspace=\newline,
    postheadhook={\textcolor{black}{\rule[.6ex]{\linewidth}{0.4pt}}\\},
    spacebelow=\parsep,
    spaceabove=\parsep,
    mdframed={
        backgroundcolor=yellow!30, 
        linecolor=yellow!90, 
        innertopmargin=6pt,
        roundcorner=5pt, 
        innerbottommargin=6pt, 
        skipbelow=\parsep, 
        skipbelow=\parsep} 
]{general}

% Example
\declaretheorem[
    style=general,
    name=Example,
    numberwithin=chapter
]{example}

% Definition
\declaretheorem[
    style=general,
    name=Definition,
    numberwithin=chapter
]{definition}

% Remark
\declaretheorem[
	style=general,
	name=Remark,
	numberwithin=chapter,
]{remark}

% Theorem
\declaretheorem[
	style=general,
	name=Theorem,
	numberwithin=chapter
]{theorem}

% Proposition
\declaretheorem[
	style=general,
	name=Proposition,
	numberwithin=chapter
]{proposition}

% Lemma
\declaretheorem[
	style=general,
	name=Lemma,
	numberwithin=chapter
]{lemma}

% Corollary
\declaretheorem[
	style=general,
	name=Corollary,
	numberwithin=chapter
]{corollary}